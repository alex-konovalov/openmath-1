% generated by GAPDoc2LaTeX from XML source (Frank Luebeck)
\documentclass[a4paper,11pt]{report}
\usepackage{a4wide}
\sloppy
\pagestyle{myheadings}
\usepackage{amssymb}
\usepackage[latin1]{inputenc}
\usepackage{makeidx}
\makeindex
\usepackage{color}
\definecolor{DarkOlive}{rgb}{0.1047,0.2412,0.0064}
\definecolor{FireBrick}{rgb}{0.5812,0.0074,0.0083}
\definecolor{RoyalBlue}{rgb}{0.0236,0.0894,0.6179}
\definecolor{RoyalGreen}{rgb}{0.0236,0.6179,0.0894}
\definecolor{RoyalRed}{rgb}{0.6179,0.0236,0.0894}
\definecolor{LightBlue}{rgb}{0.8544,0.9511,1.0000}
\definecolor{Black}{rgb}{0.0,0.0,0.0}
\definecolor{FuncColor}{rgb}{1.0,0.0,0.0}
%% strange name because of pdflatex bug:
\definecolor{Chapter }{rgb}{0.0,0.0,1.0}

\usepackage{fancyvrb}

\usepackage{pslatex}

\usepackage[pdftex=true,
        a4paper=true,bookmarks=false,pdftitle={Written with GAPDoc},
        pdfcreator={LaTeX with hyperref package / GAPDoc},
        colorlinks=true,backref=page,breaklinks=true,linkcolor=RoyalBlue,
        citecolor=RoyalGreen,filecolor=RoyalRed,
        urlcolor=RoyalRed,pagecolor=RoyalBlue]{hyperref}

% write page numbers to a .pnr log file for online help
\newwrite\pagenrlog
\immediate\openout\pagenrlog =\jobname.pnr
\immediate\write\pagenrlog{PAGENRS := [}
\newcommand{\logpage}[1]{\protect\write\pagenrlog{#1, \thepage,}}
%% were never documented, give conflicts with some additional packages


\newcommand{\GAP}{\textsf{GAP}}

\begin{document}

\logpage{[ 0, 0, 0 ]}
\begin{titlepage}
\begin{center}{\Huge \textbf{\textsf{openmath}\mbox{}}}\\[1cm]
\hypersetup{pdftitle=\textsf{openmath}}
\markright{\scriptsize \mbox{}\hfill \textsf{openmath} \hfill\mbox{}}
{\Large \textbf{\textsf{OpenMath} functionality in GAP\mbox{}}}\\[1cm]
{Version 09.03.31\mbox{}}\\[1cm]
{31 March 2009\mbox{}}\\[1cm]
\mbox{}\\[2cm]
{\large \textbf{Marco Costantini    \mbox{}}}\\
{\large \textbf{Alexander Konovalov    \mbox{}}}\\
{\large \textbf{Andrew Solomon    \mbox{}}}\\
\hypersetup{pdfauthor=Marco Costantini    ; Alexander Konovalov    ; Andrew Solomon    }
\end{center}\vfill

\mbox{}\\
{\mbox{}\\
\small \noindent \textbf{Marco Costantini    } --- Email: \href{mailto://costanti at science dot unitn dot it} {\texttt{costanti at science dot unitn dot it}}\\
 --- Homepage: \href{http://www-math.science.unitn.it/~costanti/} {\texttt{http://www-math.science.unitn.it/\texttt{\symbol{126}}costanti/}}\\
 --- Address: \begin{minipage}[t]{8cm}\noindent
 Department of Mathematics\\
 University of Trento\\
 \end{minipage}
}\\
{\mbox{}\\
\small \noindent \textbf{Alexander Konovalov    } --- Email: \href{mailto://alexk at mcs dot st-andrews dot ac dot uk} {\texttt{alexk at mcs dot st-andrews dot ac dot uk}}\\
 --- Homepage: \href{http://www.cs.st-andrews.ac.uk/~alexk/} {\texttt{http://www.cs.st-andrews.ac.uk/\texttt{\symbol{126}}alexk/}}\\
 --- Address: \begin{minipage}[t]{8cm}\noindent
 School of Computer Science\\
 University of St Andrews\\
 Jack Cole Building, North Haugh,\\
 St Andrews, Fife, KY16 9SX, Scotland\\
 \end{minipage}
}\\
{\mbox{}\\
\small \noindent \textbf{Andrew Solomon    } --- Email: \href{mailto://andrew at illywhacker dot net} {\texttt{andrew at illywhacker dot net}}\\
 --- Homepage: \href{http://www.illywhacker.net/} {\texttt{http://www.illywhacker.net/}}\\
 --- Address: \begin{minipage}[t]{8cm}\noindent
 Faculty of IT\\
 University of Technology, Sydney\\
 Broadway, NSW 2007\\
 Australia\\
 \end{minipage}
}\\
\end{titlepage}

\newpage\setcounter{page}{2}
{\small 
\section*{Abstract}
\logpage{[ 0, 0, 1 ]}
 \index{SCSCP package@\textsf{openmath} package} The \textsf{openmath} package provides an OpenMath phrasebook for \textsf{GAP}: it allows \textsf{GAP} users to import and export mathematical objects encoded in OpenMath, for the
purpose of exchanging them with other applications which are OpenMath enabled. \mbox{}}\\[1cm]
{\small 
\section*{Copyright}
\logpage{[ 0, 0, 2 ]}
 This package is distributed under GPL license and the terms of the \textsf{GAP} copyright. Additionally, it contains code developed at INRIA (copyright
INRIA), under the ESPRIT project number 24969 (OpenMath). The user may not use
the library in commercial products without seeking permission from the \textsf{GAP} group (support@gap-system.org) and the CAFE team at INRIA SA
(stephane.dalmas@sophia.inria.fr). This package may be redistributed ``as is''
together with this notice. \mbox{}}\\[1cm]
{\small 
\section*{Acknowledgements}
\logpage{[ 0, 0, 3 ]}
 On various stages the development of the \textsf{openmath} package was supported by: 
\begin{itemize}
\item European Commission through ESPRIT grant EP 24969 ``Accessing and Using Mathematical Information Electronically''; (see \href{http://web.archive.org/web/20040416013945/http://www.nag.co.uk/projects/OpenMath.html} {\texttt{http://web.archive.org/web/20040416013945/http://www.nag.co.uk/projects/OpenMath.html}}).
\item EU FP6 Programme project 026133 ``SCIEnce - Symbolic Computation Infrastructure for Europe'' (see \href{http://www.symbolic-computation.org/} {\texttt{http://www.symbolic-computation.org/}}).
\end{itemize}
 We acknowledge with gratitude this support.

 \mbox{}}\\[1cm]
\newpage

\def\contentsname{Contents\logpage{[ 0, 0, 4 ]}}

\tableofcontents
\newpage

  
\chapter{\textcolor{Chapter }{Introduction}}\label{Intro}
\logpage{[ 1, 0, 0 ]}
\hyperdef{L}{X7DFB63A97E67C0A1}{}
{
  The \textsf{GAP} package \textsf{openmath} provides an \textsf{OpenMath} phrasebook for \textsf{GAP}: it allows \textsf{GAP} users to import and export mathematical objects encoded in \textsf{OpenMath}, for the purpose of exchanging them with other applications which are \textsf{OpenMath}-enabled. Visit the OpenMath Society webpage at
\texttt{\symbol{92}}URL\texttt{\symbol{123}}http://www.openmath.org\texttt{\symbol{125}}
(you may try
\texttt{\symbol{92}}URL\texttt{\symbol{123}}http://openmath.activemath.org/\texttt{\symbol{125}}),
or the ESPRIT project webpage at
\texttt{\symbol{92}}URL\texttt{\symbol{123}}http://www.nag.co.uk/projects/OpenMath.html\texttt{\symbol{125}}
(you may try
\texttt{\symbol{92}}URL\texttt{\symbol{123}}http://web.archive.org/web/20040416013945/http://www.nag.co.uk/projects/OpenMath.html\texttt{\symbol{125}}). }

  
\chapter{\textcolor{Chapter }{\textsf{OpenMath} functionality in \textsf{GAP}}}\label{OpenMathFunctionality}
\logpage{[ 2, 0, 0 ]}
\hyperdef{L}{X7FE774BE829841A0}{}
{
  
\section{\textcolor{Chapter }{Loading the package}}\label{OpenMath}
\logpage{[ 2, 1, 0 ]}
\hyperdef{L}{X87B74F4A7E059AED}{}
{
  The package is loaded as shown below 
\begin{Verbatim}[fontsize=\small,frame=single,label=Example]
  
  gap> LoadPackage("openmath");
  -----------------------------------------------------------------------------
  Loading  openmath 09.03.31 (OpenMath functionality in GAP)
  by Marco Costantini (http://www-math.science.unitn.it/~costanti/),
     Alexander Konovalov (http://www.cs.st-andrews.ac.uk/~alexk/), and
     Andrew Solomon (http://www.illywhacker.net/).
  -----------------------------------------------------------------------------
  #I  Warning: package openmath, the program `gpipe' is not compiled.
  true
  
\end{Verbatim}
 During this, suggested packages may be loaded as well. }

 
\section{\textcolor{Chapter }{Viewing OpenMath representation of an object}}\label{OpenMath}
\logpage{[ 2, 2, 0 ]}
\hyperdef{L}{X817CD8387FEDF873}{}
{
  

\subsection{\textcolor{Chapter }{OMPrint}}
\logpage{[ 2, 2, 1 ]}\nobreak
\hyperdef{L}{X7B1815F078D8312B}{}
{\noindent\textcolor{FuncColor}{$\Diamond$\ \texttt{OMPrint({\slshape obj})\index{OMPrint@\texttt{OMPrint}}
\label{OMPrint}
}\hfill{\scriptsize (function)}}\\


 OMPrint writes the default XML OpenMath encoding of GAP object \mbox{\texttt{\slshape obj}} to the standard output. 
\begin{Verbatim}[fontsize=\small,frame=single,label=Example]
  
  gap> g := Group((1,2,3));;
  gap> OMPrint(g);
  <OMOBJ>
  	<OMA>
  		<OMS cd="group1" name="Group"/>
  		<OMA>
  			<OMS cd="permut1" name="Permutation"/>
  			<OMI> 2</OMI>
  			<OMI> 3</OMI>
  			<OMI> 1</OMI>
  		</OMA>
  	</OMA>
  </OMOBJ>
  
\end{Verbatim}
 }

 

\subsection{\textcolor{Chapter }{OMString}}
\logpage{[ 2, 2, 2 ]}\nobreak
\hyperdef{L}{X876AD5BC7A8812C3}{}
{\noindent\textcolor{FuncColor}{$\Diamond$\ \texttt{OMString({\slshape obj})\index{OMString@\texttt{OMString}}
\label{OMString}
}\hfill{\scriptsize (function)}}\\


 OMString returns a string with the default XML OpenMath encoding of GAP object \mbox{\texttt{\slshape obj}}. If used with the \texttt{noomobj} option, initial and final OMOBJ tags will be omitted. 
\begin{Verbatim}[fontsize=\small,frame=single,label=Example]
  
  gap> OMString(42);
  "<OMOBJ> <OMI> 42</OMI> </OMOBJ>"
  gap> OMString((1,2):noomobj);
  "<OMA> <OMS cd=\"permut1\" name=\"permutation\"/> <OMI> 2</OMI> <OMI> 1</OMI> </OMA>"
  
\end{Verbatim}
 }

 }

 
\section{\textcolor{Chapter }{Writing and reading OpenMath code to/from streams}}\label{OpenMath}
\logpage{[ 2, 3, 0 ]}
\hyperdef{L}{X78D84A657BEFB775}{}
{
  

\subsection{\textcolor{Chapter }{OMGetObject}}
\logpage{[ 2, 3, 1 ]}\nobreak
\hyperdef{L}{X7FBDCB3C83BBB115}{}
{\noindent\textcolor{FuncColor}{$\Diamond$\ \texttt{OMGetObject({\slshape stream})\index{OMGetObject@\texttt{OMGetObject}}
\label{OMGetObject}
}\hfill{\scriptsize (function)}}\\


 \mbox{\texttt{\slshape stream}} is an input stream (see "ref: InputTextFile", "ref: InputTextUser", "ref:
InputTextString", "ref: InputOutputLocalProcess" ) with an OpenMath object on
it. OMGetObject takes precisely one object off \mbox{\texttt{\slshape stream}} and returns it as a GAP object. Both XML and binary OpenMath encoding are
supported: autodetection is used. This function requires either that the \textsf{GAP} package \textsf{GAPDoc} is available (for XML OpenMath), or that the external program `gpipe',
included in this package, has been compiled (for both XML and binary
OpenMath). This may be used to retrieve objects from a file, for example: 
\begin{Verbatim}[fontsize=\small,frame=single,label=Example]
  
  gap> test3:=Filename(DirectoriesPackageLibrary("openmath","tst"),"test3.omt");;
  gap> stream := InputTextFile( test3 );;
  gap> OMGetObject(stream);
  912873912381273891
  gap> OMGetObject(stream);
  E(4)
  gap> CloseStream(stream);
  
\end{Verbatim}
 or it can be used to retrieve them from standard input - one may paste an
OpenMath object directly into standard input after issuing GAP with the
following commands: 
\begin{Verbatim}[fontsize=\small,frame=single,label=Example]
  
  gap> stream := InputTextUser();;
  gap> g := OMGetObject(stream);CloseStream(stream);
  
\end{Verbatim}
 }

 

\subsection{\textcolor{Chapter }{OMPutObject}}
\logpage{[ 2, 3, 2 ]}\nobreak
\hyperdef{L}{X84FC6AD5872CBF33}{}
{\noindent\textcolor{FuncColor}{$\Diamond$\ \texttt{OMPutObject({\slshape stream, obj})\index{OMPutObject@\texttt{OMPutObject}}
\label{OMPutObject}
}\hfill{\scriptsize (function)}}\\


 OMPutObject writes (appends) the XML OpenMath encoding of the GAP object \mbox{\texttt{\slshape obj}} to output stream \mbox{\texttt{\slshape stream}} (see "ref: OutputTextFile", "ref: OutputTextUser", "ref: OutputTextString",
"ref: InputOutputLocalProcess" ). 
\begin{Verbatim}[fontsize=\small,frame=single,label=Example]
  
  gap> g := [[1,2],[1,0]];;
  gap> t := "";
  ""
  gap> s := OutputTextString(t, true);;
  gap> OMPutObject(s, g);
  gap> CloseStream(s);
  gap> Print(t);
  <OMOBJ>
  	<OMA>
  		<OMS cd="linalg2" name="matrix"/>
  		<OMA>
  			<OMS cd="linalg2" name="matrixrow"/>
  			<OMI> 1</OMI>
  			<OMI> 2</OMI>
  		</OMA>
  		<OMA>
  			<OMS cd="linalg2" name="matrixrow"/>
  			<OMI> 1</OMI>
  			<OMI> 0</OMI>
  		</OMA>
  	</OMA>
  </OMOBJ>
  
\end{Verbatim}
 }

 }

 
\section{\textcolor{Chapter }{Auxiliary functions}}\label{Auxiliary}
\logpage{[ 2, 4, 0 ]}
\hyperdef{L}{X866E18057EF83F65}{}
{
  

\subsection{\textcolor{Chapter }{OMTest}}
\logpage{[ 2, 4, 1 ]}\nobreak
\hyperdef{L}{X851F202484C56D67}{}
{\noindent\textcolor{FuncColor}{$\Diamond$\ \texttt{OMTest({\slshape obj})\index{OMTest@\texttt{OMTest}}
\label{OMTest}
}\hfill{\scriptsize (function)}}\\


 Converts \mbox{\texttt{\slshape obj}} to OpenMath and back. Returns true iff \mbox{\texttt{\slshape obj}} is unchanged (as a GAP object) by this operation. The OpenMath standard does
not stipulate that converting to and from OpenMath should be the identity
function so this is a useful diagnostic tool. }

 }

 }

 \def\bibname{References\logpage{[ "Bib", 0, 0 ]}
\hyperdef{L}{X7A6F98FD85F02BFE}{}
}

\bibliographystyle{alpha}
\bibliography{manual}

\def\indexname{Index\logpage{[ "Ind", 0, 0 ]}
\hyperdef{L}{X83A0356F839C696F}{}
}


\printindex

\newpage
\immediate\write\pagenrlog{["End"], \arabic{page}];}
\immediate\closeout\pagenrlog
\end{document}
