%%%%%%%%%%%%%%%%%%%%%%%%%%%%%%%%%%%%%%%%%%%%%%%%%%%%%%%%%%%%%%%%%%%%%%%%%
%%
%W  manual.tex                GAP documentation             Thomas Breuer
%W                                                         & Frank Celler
%W                                                     & Martin Schoenert
%W                                                       & Heiko Theissen
%%
%H  $Id$
%%
%%
%%%%%%%%%%%%%%%%%%%%%%%%%%%%%%%%%%%%%%%%%%%%%%%%%%%%%%%%%%%%%%%%%%%%%%%%%
%%
%F  gapomacro . . . . . . . . . . . . . . . .  read the GAP macro package
%%
\input ../../../doc/gapmacro
%
%
%%%%%%%%%%%%%%%%%%%%%%%%%%%%%%%%%%%%%%%%%%%%%%%%%%%%%%%%%%%%%%%%%%%%%%%%%
%%
%F  BeginningOfBook . . . . . . . . . . . . . . . . . . .  start the book
%%
\BeginningOfBook{OpenMath}
%
%
%%%%%%%%%%%%%%%%%%%%%%%%%%%%%%%%%%%%%%%%%%%%%%%%%%%%%%%%%%%%%%%%%%%%%%%%%
%%
%F  UseReferences . . . . . . . . . . . . . . . . . .  specify references
%%
\UseReferences{../../../doc/ext}
\UseReferences{../../../doc/ref}
%
%
%%%%%%%%%%%%%%%%%%%%%%%%%%%%%%%%%%%%%%%%%%%%%%%%%%%%%%%%%%%%%%%%%%%%%%%%%
%%
%F  TitlePage . . . . . . . . . . . . . . . . . . . . . . nice title page
%%
\TitlePage{
  \centerline{\titlefont OpenMath}
  \centerline{\secfont version 06.03.02}
		\medskip
		\medskip
		\vfill
  \centerline{\secfont Andrew Solomon}
    \medskip
		\medskip
  \centerline{\secfont Department of Computer Science}
    \medskip
  \centerline{\secfont University of St.~Andrews, North Haugh, St.~Andrews,
              Fife KY16 9SS, Scotland}
		\medskip
This work is funded by the European Commission
through ESPRIT grant EP 24969
``Accessing and Using Mathematical Information Electronically''

		\vfill
\centerline{and}
		\medskip
\centerline{\secfont Marco Costantini}
}
%
%
\Colophon{
\input ../README
}
%
%
%%%%%%%%%%%%%%%%%%%%%%%%%%%%%%%%%%%%%%%%%%%%%%%%%%%%%%%%%%%%%%%%%%%%%%%%%
%%
%F  TableOfContents . . . . . . . . . . . .  generate a table of contents
%%
\TableOfContents

%
%
%%%%%%%%%%%%%%%%%%%%%%%%%%%%%%%%%%%%%%%%%%%%%%%%%%%%%%%%%%%%%%%%%%%%%%%%%
%%
%F  FrontMatter . . . . . . . . . . . . . . . . . . . . .  GAP 4 Tutorial
%%
\FrontMatter
\immediate\write\citeout{\bs bibdata{mrabbrev,manual}}

\Chapter{A short introduction to OpenMath}

{\it From the official OpenMath society website:}

OpenMath is an emerging standard for representing mathematical objects with their semantics, allowing them to be
exchanged between computer programs, stored in databases, or published on the worldwide web. While the original
designers were mainly developers of computer algebra systems, it is now attracting interest from other areas of scientific
computation and from many publishers of electronic documents with a significant mathematical content. There is a strong
relationship to the MathML recommendation from the Worldwide Web Consortium, and a large overlap between the two
developer communities. MathML deals principally with the presentation of mathematical objects, while OpenMath is solely
concerned with their semantic meaning or content. While MathML does have some limited facilities for dealing with content,
it also allows semantic information encoded in OpenMath to be embedded inside a MathML structure. Thus the two
technologies may be seen as highly complementary. 

Mathematical objects encoded in OpenMath can be:

\noindent 
$\bullet$ displayed in a browser 

\noindent
$\bullet$      exchanged between software systems 

\noindent
$\bullet$      cut and pasted for use in different contexts 

\noindent
$\bullet$  verified as being mathematically sound (or not!) 

\noindent
$\bullet$     used to make interactive documents really interactive. 

OpenMath is highly relevant for persons working with mathematics on computers, for those working with large documents
(e.g. databases, manuals) containing mathematical expressions, and for technical and mathematical publishing. 

The worldwide OpenMath activities are coordinated within the OpenMath Society, based in Helsinki, Finland. It is
coordinated by an executive committee, elected by its members. It organizes workshops and holds an annual meeting. The
Society brings together tool builders, software suppliers, publishers and authors. 

{\it  OpenMath and {\GAP}}

This package provides an OpenMath phrasebook for {\GAP}: 
it allows {\GAP} users to 
import and export mathematical objects encoded in OpenMath, 
for the purpose of exchanging them with other applications which 
are OpenMath enabled.

{\it  Further Information}

Visit the OpenMath Society webpage at \URL{http://www.openmath.org}
(you may try \URL{http://openmath.activemath.org/}), or
the ESPRIT project webpage at \URL{http://www.nag.co.uk/projects/OpenMath.html}
(you may try 
\URL{http://web.archive.org/web/20040416013945/http://www.nag.co.uk/projects/OpenMath.html}).




%
%%%%%%%%%%%%%%%%%%%%%%%%%%%%%%%%%%%%%%%%%%%%%%%%%%%%%%%%%%%%%%%%%%%%%%%%%
%%
%F  Chapters  . . . . . . . . . . . . . . . . . . . . . .  GAP 4 Tutorial
%%
\Chapters
\Input{openmath}
%
%
%%%%%%%%%%%%%%%%%%%%%%%%%%%%%%%%%%%%%%%%%%%%%%%%%%%%%%%%%%%%%%%%%%%%%%%%%
%%
%F  Appendices  . . . . . . . . . . . .  Extending GAP 4 Reference Manual
%%
\Appendices
%\Bibliography
%\Index
%
%
%%%%%%%%%%%%%%%%%%%%%%%%%%%%%%%%%%%%%%%%%%%%%%%%%%%%%%%%%%%%%%%%%%%%%%%%%
%%
%F  EndOfBook . . . . . . . . . . . . . . . . . . . . . . . . . that's it
%%
\EndOfBook
%
%
%%%%%%%%%%%%%%%%%%%%%%%%%%%%%%%%%%%%%%%%%%%%%%%%%%%%%%%%%%%%%%%%%%%%%%%%%
%%
%E  manual.tex	. . . . . . . . . . . . . . . . . . . . . . . . ends here
